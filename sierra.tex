\documentclass[10pt]{amsart}
\usepackage[skins,minted]{tcolorbox}
\definecolor{mintedbackground}{rgb}{0.95,0.95,0.95}

\setminted[rust]{
    bgcolor=mintedbackground,
    fontfamily=tt,
    linenos=true,
    numberblanklines=true,
    numbersep=12pt,
    numbersep=5pt,
    gobble=0,
    frame=leftline,
    framesep=2mm,
    funcnamehighlighting=true,
    tabsize=4,
    obeytabs=false,
    mathescape=false
    samepage=false,
    showspaces=false,
    showtabs =false,
    texcl=false,
    baselinestretch=1.2,
    breaklines=true,
}
\newtcblisting{code}[2][]{listing engine=minted, listing only, #1, title=\texttt{#2}, fonttitle=\tiny, minted language=rust, colback=mintedbackground, minted options={fontsize=\tiny}}

\begin{document}
\title{On Sierra}
\author{Malcolm Langfield}
\begin{abstract}
  We look at a program in high-level Cairo, Sierra, and Casm.
\end{abstract}

\maketitle

Consider the following example program written in Cairo 1.0:
\vspace{1em}
\begin{code}{cairo/examples/corelib\_usage.cairo}
impl MyCopy of Copy::<Option<(felt252, felt252)>>;

fn foo(x: Option<(felt252, felt252)>) -> Option<felt252> {
    let y = x;
    match x {
        Option::Some(x) => {
            let (x, y) = x;
            Option::Some(x)
        },
        // TODO(spapini): Replace with _.
        Option::None(o) => {
            return Option::None(());
        },
    }
}
\end{code}
This is a simple function which unpacks the first element of a $2$-tuple from within an \texttt{Option} monad. Let us look at how this is represented in Sierra code:

\begin{code}{cairo/tests/test\_data/corelib\_usage.sierra}
type felt252 = felt252;
type Tuple<felt252, felt252> = Struct<ut@Tuple, felt252, felt252>;
type Unit = Struct<ut@Tuple>;
type core::option::Option::<(core::felt252, core::felt252)> = Enum<ut@core::option::Option::<(core::felt252, core::felt252)>, Tuple<felt252, felt252>, Unit>;
type core::option::Option::<core::felt252> = Enum<ut@core::option::Option::<core::felt252>, felt252, Unit>;

libfunc enum_match<core::option::Option::<(core::felt252, core::felt252)>> = enum_match<core::option::Option::<(core::felt252, core::felt252)>>;
libfunc branch_align = branch_align;
libfunc struct_deconstruct<Tuple<felt252, felt252>> = struct_deconstruct<Tuple<felt252, felt252>>;
libfunc drop<felt252> = drop<felt252>;
libfunc enum_init<core::option::Option::<core::felt252>, 0> = enum_init<core::option::Option::<core::felt252>, 0>;
libfunc store_temp<core::option::Option::<core::felt252>> = store_temp<core::option::Option::<core::felt252>>;
libfunc drop<Unit> = drop<Unit>;
libfunc struct_construct<Unit> = struct_construct<Unit>;
libfunc enum_init<core::option::Option::<core::felt252>, 1> = enum_init<core::option::Option::<core::felt252>, 1>;

enum_match<core::option::Option::<(core::felt252, core::felt252)>>([0]) { fallthrough([1]) 7([2]) };
branch_align() -> ();
struct_deconstruct<Tuple<felt252, felt252>>([1]) -> ([3], [4]);
drop<felt252>([4]) -> ();
enum_init<core::option::Option::<core::felt252>, 0>([3]) -> ([5]);
store_temp<core::option::Option::<core::felt252>>([5]) -> ([6]);
return([6]);
branch_align() -> ();
drop<Unit>([2]) -> ();
struct_construct<Unit>() -> ([7]);
enum_init<core::option::Option::<core::felt252>, 1>([7]) -> ([8]);
store_temp<core::option::Option::<core::felt252>>([8]) -> ([9]);
return([9]);

examples::corelib_usage::foo@0([0]: core::option::Option::<(core::felt252, core::felt252)>) -> (core::option::Option::<core::felt252>);
\end{code}



\end{document}
